\chapter{Facial Recognition}\label{ch:face-rec}
Facial recognition is a task of verifying or identifying a person from digital image/video.

As I mentioned in the definition, there are two main subtasks~\cite{FaceRec}:
\begin{enumerate}
    \item \textbf{Verification} deals with verifying whether the person in the image is who he claims he is.
    Typical modern use case of verification is smartphone unlocking with face.
    An example of such system is Face ID developed by Apple Inc.

    \item \textbf{Identification} is a task of matching a person to an identity.
    To formulate it in another way, the goal of identification is to give us an answer to the question of who the person
    in the image is.
\end{enumerate}

\section{Pipeline}\label{sec:pipeline}

Facial recognition pipeline\footnote{A chain of processing elements, arranged so that the output of each element is the
input of the next.} usually has the following four steps:
\begin{enumerate}
    \item \textbf{Face detection} deals with determination of face location within the image.
    The output of the algorithm is usually face coordinates and facial landmarks.
    Landmarks are a set of coordinates marking important points of the face (eyebrows, nose, mouth, \ldots).
    The knowledge of these points is a necessity for the following step.
    \item \textbf{Face Alignment} is a task of changing the face position in such a way that it resembles the position
    of faces on which the Feature Extraction model was trained.
    In most of the instances this step improves the accuracy.
    \item \textbf{Feature Extraction} is a process of computing a feature vector\footnote{A feature vector is a vector
    that contains information describing an object's important characteristics.} of the face.
    Modern methods use CNNs to accomplish this task.
    \item \textbf{Feature Matching} uses feature vector from the previous task to identify the person in the image.
    The algorithm uses a database of pre-computed feature vectors and compares them to the newly extracted one.
    The identity associated with the feature vector which has the smallest distance from the extracted one is
    considered to be the identity of the person in the image.
\end{enumerate}