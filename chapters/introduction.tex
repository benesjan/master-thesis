\chapter{Introduction}\label{ch:introduction}
Facial recognition systems recently exceeded the performance of humans on many real world benchmarks.

The beginning of the quest to give computers the ability to recognize human faces dates back to 1960s~\cite{History}
with Woody Bladson being the first researcher to attempt the feat.
The issue with the first systems was the decrease of performance when the faces was not in the optimal position.
Big stepping stone towards pose invariance came with the invention of convolutional neural networks~\ref{ch:cnn}.
These models are the cornerstone of modern computer vision.

In the field of facial recognition a lot of effort was put into the design of loss functions.
These functions are essential in all of machine learning, for they are used to supervise the model fitting.
The researcher's goal is to design this function in such a way that mathematical optimization leads to the model
with desired properties.
In the last few years the research was very successful which is the reason why the system now achieves superhuman
abilities.
An impressive property of these systems is the grace with which they generalize beyond faces present in the training
dataset.

\section{Objectives}\label{sec:objectives}
The objectives of this text are:
\begin{enumerate}
    \item to provide an overview of modern facial recognition methods;
    \item to implement the state-of-the-art algorithm;
    \item to evaluate the system's performance on appropriate benchmark dataset;
    \item to compare the results with commercial algorithm.
\end{enumerate}

\section{Outline}\label{sec:outline}