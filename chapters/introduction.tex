\chapter{Introduction}\label{ch:introduction}
Facial recognition systems recently exceeded the performance of humans on many real world benchmarks.

The beginning of the quest to give computers the ability to recognize human faces dates back to 1960s~\cite{History}
with Woody Bladson being the first researcher to attempt the feat.
The issue with the first systems was the decrease of performance when the faces were not in the optimal position.
Big stepping stone towards pose invariance came with the invention of convolutional neural networks~\ref{ch:cnn}.
These models are the cornerstone of modern computer vision.

In the field of facial recognition a lot of effort was put into the design of loss functions.
These functions are essential in all of machine learning, for they are used to supervise model fitting.
The researcher's goal is to design this function in such a way that mathematical optimization leads to the model
with desired properties.
In the last few years, as is manifested in the superhuman performance of these models, the research was very successful.
An impressive property of these systems is the grace with which they generalize beyond faces present in the training
dataset.
This makes it really easy to determine that the person is not in the set of known identities.
This is crucial for biometric authentication.

The improved accuracy and reliability of facial recognition technology led to a wide spread deployment of these systems.
These systems found their place in our phones giving them the ability to automatically unlock;
they found place on the servers of big technology companies, like Google who uses the technology to better organize
our photos;
the technology is also at the heart of modern surveillance systems.

The last mentioned use case is the reason why the deployment of these algorithms is plagued by controversy.
As the saying goes, technology is a double edged sword.
Facial recognition is the embodiment of this saying.
On the positive side there were many cases in which the technology helped authorities capture dangerous
individuals~\cite{FacRecCon}.
However, no matter how good this benefit is, for many it is not worth the risk of losing the privacy.
This is the case in San Francisco, where the legislators voted to ban the use of the technology by
law enforcement~\cite{FacRecSan}.


\section{Objectives}\label{sec:objectives}
The objectives of this text are:
\begin{enumerate}
    \item to provide an overview of modern facial recognition methods;
    \item to implement the state-of-the-art algorithm;
    \item to evaluate the system's performance on appropriate benchmark dataset;
    \item to compare the results with commercial algorithm.
\end{enumerate}

\section{Outline}\label{sec:outline}