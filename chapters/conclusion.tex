\chapter{Conclusion}\label{ch:conclusion}
The main topic of this thesis is automatic face recognition.

In the introduction~\ref{sec:goals} I stated four goals of this study.

First goal was to provide an overview of modern facial recognition methods.
I split this overview into three parts.
The first of these is a chapter about convolutional neural networks~\ref{ch:cnn} (CNNs), which are the building
block of modern facial recognition systems.
The second one is a chapter called Facial Recognition~\ref{ch:face-rec}.
In this part I describe how the end-to-end facial recognition system works.
The last part (Loss Functions\ref{ch:loss-functions}) focuses on the research of loss functions which perform
especially well in facial recognition tasks.

Second goal was to implement a state-of-the-art algorithm.
The description of the algorithm is in the chapter~\ref{ch:implementation}.
The implementation utilizes the ArcFace research~\ref{sec:arcface}.

Third and fourth goal was to evaluate my algorithm's performance and compare the results with commercial one
(EyeFace SDK~\ref{subsec:eyeface}).
My algorithm exceeded the commercial system by two percents~\ref{subsec:syseval}.

To sum up the thesis, the assumption that open-source and freely available research and algorithms exceed
their commercial counterparts has been proven correct.
Even though the results are good, there is still a room for improvement (see section~\ref{sec:possible-improvements}).
\newpage

\section{Possible Improvements}\label{sec:possible-improvements}
I have two ideas which could lead to the improvements in system's performance:
\begin{itemize}
    \item \textit{Fitting the model on relevant identities.} The model used at the core of the system (ResNet-18) was
    fitted on identities which are not relevant in the context of the system's application.
    While this is not an issue as the model works well on open-set\footnote{See footnote on page~\pageref{foot:openset}}
    problems, fine-tuning\footnote{A process of taking the weights of a trained neural network and using them as
    an initialization for a new model being trained on a data from the same domain.} the model again is one way to
    improve the system.
    \item \textit{Better face detection engine.} An area where the system is lacking is the face detection engine.
    The detector available in the EyeFace SDK~\ref{subsec:eyeface} seems to be significantly more robust than the one
    used in this work.
    This manifested itself when looking at the images of faulty MTCNN detection (see~\ref{fig:faulty_bbox}).
    EyeFace detector is better at handling edge cases (e.g. white stripe over part of the face).
    Such edge cases lead to MTCNN producing false negatives.
\end{itemize}