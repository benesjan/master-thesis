\chapter{Conclusion}\label{ch:conclusion}
In the introduction~\ref{sec:goals} I stated goals of this study.

The main objective was to implement state-of-the-art algorithm and compare its performance with a commercial system.
Given that the performance of my system exceeded that of commercial one by two percents I dare to say that the
goal of this thesis was accomplished.

\section{Possible Improvements}\label{sec:possible-improvements}
I have two ideas which could lead to the improvements in system's performance:
\begin{itemize}
    \item \textit{Fitting the model on relevant identities.} The model used at the core of the system (ResNet-18) was
    fitted on identities which are not relevant in the context of the system's application.
    While this is not an issue as the model works well on open-set\footnote{See footnote on page~\pageref{foot:openset}}
    problems, training the model again is one way to improve the system.
    \item \textit{Better face detection engine.} An area where the system is lacking is the face detection engine.
    The detector available in the Eyedea SDK seems to be significantly more robust than the one used in this work.
    This manifested itself when looking at the images of faulty MTCNN detection (see~\ref{fig:faulty_bbox}).
    Eyedea detector is better at handling edge cases (e.g. white stripe over part of the face).
    Such edge cases lead to MTCNN producing false negatives.
\end{itemize}