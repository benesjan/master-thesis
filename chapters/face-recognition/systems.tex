\section{Commercial Systems}\label{sec:systems}
There are two main approaches of training CNNs for face recognition.

The first one is to train a multi-class classifier which can separate identities directly.
An example of such system is DeepFace~\ref{subsec:deepface}.

The second approach is to learn embedding using a triplet loss~\ref{sec:triplet-loss} function or similar.
FaceNet~\ref{subsec:facenet} is a known example of a system being trained using the second approach.

\subsection{DeepFace}\label{subsec:deepface}
DeepFace~\cite{DeepFace} is a system developed by FaceBook Inc. in 2014.

The research is notable for its use of advanced alignment technique which consists of three steps:

\begin{enumerate}
    \item The first step is \textbf{2D Alignment} and it begins with detection of 6 fiducial points/landmarks.
    These points and the reference position of points are then used to find a transformation.
    This transformation then generates 2D aligned and cropped image.
    \item The second step is \textbf{3D Alignment}.
    In this step the image is warped onto a generic 3D shape model.
    This is achieved by localization of 67 fiducial points in the image and then fitting an affine
    camera\footnote{linear mathematical model to approximate the perspective projection followed by an ideal
    pinhole camera.} \textit{P} using the generalized least squares solution and the reference position $x_{3d}$ of
    points on the 3D shape model.
    \item \textbf{Frontalization} is the final step and it is achieved by a piece-wise affine transformation T from
    $x_{2d}$ source to $\tilde{x_{3d}}$ target.
    The target $\tilde{x_{3d}}$ is a list of positions of reference fiducial points from previous step enriched with
    residuals \textit{r}.
    These residuals were added to the reference positions $\tilde{x_{3d}}$ to account for non-rigid deformations which
    are not modeled by the affine camera \textit{P}.
    Without these residuals, all faces would be warped into the same shape losing important discriminative factors.
\end{enumerate}

\begin{figure}[H]
    \centering
    \includegraphics[width=\columnwidth]{images/face-recognition/deepface.png}
    \caption{Outline of DeepFace architecture~\cite{DeepFace}}
    \label{fig:deepface}
\end{figure}

There are 9 layers in the model with over 120 million parameters.
The process of classification is visualized in the picture~\ref{fig:deepface}.
The model was trained on more than 4 million images and as the name of the research paper~\cite{DeepFace} implies,
the results (\textbf{97.35\%} on LFW dataset~\ref{subsec:lfw}) almost matched the results of humans (\textbf{97.53\%}
on LFW dataset).


\subsection{FaceNet}\label{subsec:facenet}
FaceNet~\cite{FaceNet} is a system developed by researchers from Google Inc. in 2015.

Interesting innovation of FaceNet lies in the format of its output.
The output of the network is a vector representing a position in an euclidean space (so called embeddings) instead of a
number representing identity.
This approach allows for straight-forward implementation of \textit{verification} and
\textit{identification}~\ref{ch:face-rec}.
Implementation of verification involves thresholding the distance between two embeddings; and identification becomes
k-NN classification problem.

\begin{figure}[H]
    \centering
    \includegraphics[width=\columnwidth]{images/face-recognition/facenet.png}
    \caption{Outline of FaceNet architecture~\cite{FaceNet}}
    \label{fig:facenet}
\end{figure}

The loss function used to train the model is called \textit{triplet loss}~\ref{sec:triplet-loss}.
Researches at Google came up with new online method\footnote{Training samples are selected during training.} of which
ensures rising difficulty of triplets as the network trains.

The advantages of the model are its accuracy and the compactness of the face representation.
The accuracy of the model exceeded that of human with \textbf{99.63\%} on LFW dataset~\ref{subsec:lfw} and the
euclidean space has only 128 dimensions.
Another advantage is that the model achieved great results on faces which are not in ideal position without using any
of the complex 3D modeling techniques (as was the case in DeepFace system~\ref{subsec:deepface}).
To use proper terms the system is \textit{pose invariant}.