\chapter{Facial Recognition}\label{ch:face-rec}
Facial recognition is a task of verifying or identifying a person from digital image/video.

As I mentioned in the definition, there are two main subtasks~\cite{FaceRec}:
\begin{enumerate}
    \item \textbf{Verification} deals with verifying whether the person in the image is who he claims he is.
    Typical modern use case of verification is smartphone unlocking with face.
    An example of such system is Face ID developed by Apple Inc.

    \item \textbf{Identification} is a task of matching a person to an identity.
    To formulate it in another way, the goal of identification is to give us an answer to the question of who the person
    in the image is.
\end{enumerate}

\section{Pipeline}\label{sec:pipeline}

Facial recognition pipeline\footnote{\label{foot:pipe}A chain of processing elements, arranged so that the output of
each element is the input of the next.} usually has the following four steps:
\begin{enumerate}
    \item The first one is \textbf{face detection}.
    As the name implies it deals with determination of face location within the image.
    The output of the algorithm is usually face coordinates and facial landmarks.
    Landmarks are a set of coordinates marking important points of the face (eyebrows, nose, mouth, \ldots).
    The knowledge of these points is a necessity for the following step.
    \item \textbf{Face alignment} is a task of changing the face position in such a way that it resembles the position
    of faces on which the feature extraction model was trained.
    In most of the instances this step improves the accuracy.
    \item \textbf{Feature extraction} is a process of computing a feature vector\footnote{A feature vector is a vector
    that contains information describing an object's important characteristics.} of the face.
    Architecture of models used for feature extraction was described in the previous chapter~\ref{ch:cnn}.
    \item \textbf{Feature matching} uses feature vector from the previous task to identify the person in the image.
    The algorithm uses a database of pre-computed feature vectors and compares them to the newly extracted one.
    The identity associated with the feature vector which has the smallest distance from the extracted one is
    considered to be the identity of the person in the image.
\end{enumerate}

It is important to note that the second step is not always present and it is deemed unnecessary by some~\cite{FaceNet}.

\section{Datasets}\label{sec:datasets}
In this section I will briefly describe datasets used for training and evaluation of facial recognition models.
There are too many different datasets used in practice.
For this reason I will focus only on those mentioned in this text.

\subsection{LFW}\label{subsec:lfw}
LFW is an acronym for Labeled Faces in the Wild.
The dataset contains 13,000 images and 1680 identities.
Every identity is represented by at least two samples.
The faces were detected by Viola-Jones face detector\footnote{Real-time object detection framework.}.

There are now four publicly used versions of the dataset.
These versions are differentiated by the type of preprocessing (different methods of alignment) applied to the images.

\subsection{YTF}\label{subsec:ytf}
YTF stands for YouTube Faces.
The data set contains 3425 videos and 1,595 unique identities.
The average length of the video clip is 181.3 frames and there are on average 2.15 videos for each subject.

\subsection{MS-Celeb-1M}\label{subsec:ms1m}
MS-Celeb-1M is a dataset constructed by Microsoft Research.
There are 10 million face images with nearly 100,000 individuals.
The data were harvested from the Internet.

Due to the method with which the images were collected there are many mislabellings in the dataset.
For this reason there are different versions available on the Internet containing refined data (like MS1MV2).

As the name implies the dataset contains images of celebrities.
In this context celebrity is assumed to be anyone with frequent online presence.
This became a controversial issue and as a result Microsoft pulled the dataset off the internet.

\section{Face Recognition Systems}\label{sec:systems}
There are two main approaches of training CNNs for face recognition.

The first one is to train a multi-class classifier which can separate identities directly.
An example of such system is DeepFace~\ref{subsec:deepface}.

The second approach is to learn embedding using the triplet loss~\ref{sec:triplet-loss} function or similar.
FaceNet~\ref{subsec:facenet} is an example of a system being trained using the second approach.

\subsection{DeepFace}\label{subsec:deepface}
DeepFace~\cite{DeepFace} is a system developed by FaceBook Inc. in 2014.

The research is notable for its use of advanced alignment technique which consists of three steps:

\begin{enumerate}
    \item \textbf{2D Alignment}

    In this step, the image is aligned in such a way that the fiducial points/landmarks are in a similar position
    to predetermined reference positions.
    To carry out this process it is first necessary to detect the 6 fiducial points/landmarks.
    These points and the reference positions are then used to find the parameters of an affine transformation.
    Applying the transformation to the original image gives us the desired result.
    \item \textbf{3D Alignment}

    In the second step, the image is warped onto a generic 3D shape model.
    This is achieved by localization of 67 fiducial points in the image and then fitting an affine
    camera\footnote{linear mathematical model to approximate the perspective projection followed by an ideal
    pinhole camera.} \textit{P} using the generalized least squares solution and the reference position $x_{3d}$ of
    points on the 3D shape model.

    \item \textbf{Frontalization}

    This is the final step and it consist of a computation and application of a piece-wise affine transformation T from
    $x_{2d}$ source to $\tilde{x_{3d}}$ target.
    The target $\tilde{x_{3d}}$ is a list of positions of reference fiducial points from the previous step enriched with
    residuals \textit{r}.
    These residuals were added to the reference positions $\tilde{x_{3d}}$ to account for non-rigid deformations which
    are not modeled by the affine camera \textit{P}.
    Without these residuals, all faces would be warped into the same shape losing important discriminative factors.
\end{enumerate}

\begin{figure}[H]
    \centering
    \includegraphics[width=\columnwidth]{images/face-recognition/deepface.png}
    \caption{Outline of DeepFace architecture~\cite{DeepFace}}
    \label{fig:deepface}
\end{figure}

There are 9 layers in the model with over 120 million parameters.
The process of classification is visualized in the picture~\ref{fig:deepface}.
The model was trained on more than 4 million images and as the name of the research paper~\cite{DeepFace} implies,
the results (\textbf{97.35\%} on LFW dataset~\ref{subsec:lfw}) almost matched the results of humans (\textbf{97.53\%}
on LFW dataset).


\subsection{FaceNet}\label{subsec:facenet}
FaceNet~\cite{FaceNet} is a system developed by researchers at Google Inc. in 2015.

An interesting innovation of FaceNet is the format of its output.
The output of the network is a vector representing a position in an euclidean space (so called embeddings) instead of a
number representing an identity.
This approach allows for straight-forward implementation of \textit{verification} and
\textit{identification}~\ref{ch:face-rec}.
The implementation of verification involves thresholding the distance between the reference and the newly obtained
embedding; and identification becomes k-NN classification problem.

\begin{figure}[H]
    \centering
    \includegraphics[width=\columnwidth]{images/face-recognition/facenet.png}
    \caption{Outline of FaceNet architecture~\cite{FaceNet}}
    \label{fig:facenet}
\end{figure}

The loss function used to train the model is called \textit{triplet loss}~\ref{sec:triplet-loss}.
Researches at Google came up with a new online method\footnote{Training samples are selected during training.} which
ensures that the difficulty of triplets is rising as the network trains.

The advantages of the model are its accuracy and the compactness of the face representation.
The accuracy exceeded that of human with \textbf{99.63\%} on LFW dataset~\ref{subsec:lfw} and the euclidean space has
only 128 dimensions.

Another advantage is how well the model handles faces which are not in ideal position.
This removed the need for complex preprocessing and face frontalization.
To use proper terms the system is \textit{pose-invariant}.

\subsection{EyeFace SDK}\label{subsec:eyeface}
EyeFace SDK is a library providing face detection, face recognition, etc. developed by Eyedea Recognition s. r. o..
One of the goals of this thesis is to exceed the performance of this commercial algorithm.
The SDK is closed source; therefore, the implementation details are not known.